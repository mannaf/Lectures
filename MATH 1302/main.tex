\documentclass{beamer}
\usepackage{amsmath,amssymb}
\usepackage{pgfplots}
\pgfplotsset{compat=1.18}
\usepgfplotslibrary{fillbetween}
\usetikzlibrary{calc}
\usepackage{tcolorbox}
\usetheme{Madrid}

% Title Information
\title{Complex Analysis}
\subtitle{ Theory and Math}
\author{Md. Abde Mannaf}
\institute{Senior Lecturer, NUB}
\date{\today}

\begin{document}

%---------------------------------------------
% Title Slide
\frame{\titlepage}

% Outline Slide
\begin{frame}
\frametitle{Presentation Outline}
\tableofcontents
\end{frame}

%---------------------------------------------
\section{Introduction}
\begin{frame}{Introduction}
    \begin{block}{Overview}
        This lecture discusses two important types of first-order differential equations:
        \begin{itemize}
            \item Linear Differential Equations
            \item Exact Differential Equations
        \end{itemize}
        These equations have specific solution techniques based on their structure.
    \end{block}
\end{frame}

%---------------------------------------------
\section{Definitions}

\begin{frame}{Definitions}
    \begin{block}{Complex number}
    Any number of the form $x+iy$ is called a complex number where $x,y \in R$. A complex number defined by $z=x+iy$
    \end{block}\pause
        
    \begin{block}{Complex variable} 
    A variable which can take any complex number number is called complex variable. The complex variable of a complex number represented by $ z =x+iy$.
    \end{block}\pause
    \begin{block}{Conjugate complex number} 
    Any number of the form $x-iy$ is called a complex number where $x,y \in R$. A complex number defined by $\bar{z} = x-iy$.
    \end{block}
\end{frame}

\begin{frame}{Definitions}
    \begin{block}{Analytic Function}
        A complex function $f(z)$ is said to be analytic at a point $z_0$ if its derivative exists not only at $z_0$ but also at each point $z$ in some neighborhood of $z_0$.
\end{block}\pause
    
\begin{block}{Cauchy's Integral Formula} 
If $f$ is analytic on a simply connected domain $D$ and $\gamma$ is a positively oriented, simple closed contour lying in $D$, then for any point $z$ inside $\gamma$,
\[
f(z) = \frac{1}{2\pi i} \int_{\gamma} \frac{f(\zeta)}{\zeta - z}\, d\zeta.
\]

    \end{block}
\end{frame}



% \begin{frame}{Definitions}
%     \begin{block}{Analytic Function}
%        \textbf{Definition (Analytic Function).}
% A function $f$ is said to be \emph{analytic} at a point $z_0$ if it can be represented by a convergent power series in some neighborhood of $z_0$; that is, if there exists $r > 0$ such that
% \[
% f(z) = \sum_{n=0}^{\infty} a_n (z - z_0)^n \quad \text{for all } |z - z_0| < r.
% \]
% A function is called \emph{analytic on a set} $D$ if it is analytic at every point of $D$.
% \end{block}\pause
    
% \begin{block}{Cauchy's Integral Formula} 
%  \textbf{Cauchy's Integral Formula.}
% If $f$ is analytic on a simply connected domain $D$ and $\gamma$ is a positively oriented, simple closed contour lying in $D$, then for any point $z$ inside $\gamma$,
% \[
% f(z) = \frac{1}{2\pi i} \int_{\gamma} \frac{f(\zeta)}{\zeta - z}\, d\zeta.
% \]

%     \end{block}
% \end{frame}









 

\begin{frame}{Theorem}
\begin{block}
If $f(z) = u(x,y) + iv(x,y)$ is analytic in a region $R$, and if $u$ and $v$
have continuous second-order partial derivatives in $R$, then both $u$ and $v$ 
are harmonic in $R$.
\end{block}
\textbf{Proof.}
Since $f$ is analytic in $R$, the real and imaginary parts $u$ and $v$ satisfy
the Cauchy--Riemann equations:
\[
u_x = v_y, \qquad u_y = -v_x.
\]
Because $u$ and $v$ have continuous second-order partial derivatives, we may
differentiate the Cauchy--Riemann equations again.

Differentiate $u_x = v_y$ with respect to $x$:
\[
u_{xx} = v_{yx}.
\]
Differentiate $u_y = -v_x$ with respect to $y$:
\[
u_{yy} = -v_{xy}.
\]

\end{frame}
\begin{frame}{continued}
    Since mixed partial derivatives are equal ($v_{xy} = v_{yx}$), we obtain
\[
u_{xx} + u_{yy}
= v_{yx} - v_{xy}
= 0.
\]
Thus,
\[
\Delta u = u_{xx} + u_{yy} = 0,
\]
which shows that $u$ is harmonic in $R$.

Similarly, differentiate $u_x = v_y$ with respect to $y$:
\[
u_{xy} = v_{yy}.
\]
Differentiate $u_y = -v_x$ with respect to $x$:
\[
u_{yx} = -v_{xx}.
\]
Again using $u_{xy} = u_{yx}$, we get
\[
v_{yy} = -v_{xx},
\]
or equivalently,
\[
v_{xx} + v_{yy} = 0.
\]
\end{frame}

\begin{frame}{Continued}
    Thus,
\[
\Delta v = v_{xx} + v_{yy} = 0,
\]
so $v$ is harmonic in $R$.

Hence, both $u$ and $v$ are harmonic functions in the region $R$.
% \quad \blacksquare
\end{frame}








\begin{frame}{Show that the function $u = 2x - x^{3} + 3xy^{2}$ is harmonic and also find the harmonic
conjugate if $\; f(z)=u+iv$ is analytic.}
  

\textbf{Solution:}

Given that, $u = 2x - x^{3} + 3xy^{2}$

\[
\therefore \frac{\partial u}{\partial x} = 2 - 3x^{2} + 3y^{2} = \Phi_{1}(x,y) \qquad \text{........(1)}
\]

\[
\frac{\partial u}{\partial y} = 6xy = \Phi_{2}(x,y) \qquad \text{........(2)}
\]

\[
\frac{\partial^{2} u}{\partial x^{2}} = -6x \qquad \text{........(3)}
\]

\[
\frac{\partial^{2} u}{\partial y^{2}} = 6x \qquad \text{........(4)}
\]

Adding equation (3) \& (4) we get,

\[
\frac{\partial^{2} u}{\partial x^{2}} + \frac{\partial^{2} u}{\partial y^{2}}
= -6x + 6x = 0
\]

This shows that $u$ satisfy Laplace equation and hence $u$ is harmonic.

\textit{To find the harmonic conjugate $v$, we have by putting $x = z$ and $y = 0$ in (1) \& (2) we get}

\[
\Phi_{1}(z,0) = 2 - 3z^{2}
\]

\[
\Phi_{2}(z,0) = 0
\]

By Milne's theorem we have

\[
f'(z) = \Phi_{1}(z,0) - i\Phi_{2}(z,0)
\]

\[
= 2 - 3z^{2} - 0
\]

\[
\therefore f(z) = \int (2 - 3z^{2})\, dz
\]

\[
= 2z - z^{3} + c
\]

\[
\Rightarrow u + iv = 2(x+iy) - (x+iy)^{3} + c
\]

\[
= 2(x+iy) - (x^{3} + 3ix^{2}y + 3i^{2}xy^{2} + i^{3}y^{3}) + c
\]

\[
= 2(x+iy) - x^{3} - 3ix^{2}y + 3xy^{2} - iy^{3} + c
\]

\[
= (2x - x^{3} + 3xy^{2}) + i(2y - 3x^{2}y + y^{3} + c_{1})
\quad ; \text{where } ic_{1} = c
\]

Equating imaginary parts we have

\[
v = 2y - 3x^{2}y + y^{3} + c_{1} \quad \text{(Ans).}
\]

\begin{flushright}
Md Abde Mannaf
\end{flushright}

\end{frame}

\begin{frame}{Example 2: Linear Equation(continued )}
    Previously :
    \[
    \frac{d}{dx}\left(\frac{y}{x}\right) = x
    \] 
    Integrate both sides:
    \[
    \frac{y}{x} = \frac{x^2}{2} + C
    \]\pause
    \[
    \Rightarrow y = \frac{x^3}{2} + Cx
    \]
\end{frame}
\section{Bernoulli Differential equation}
% -------------------------------
\begin{frame}{Bernoulli Equation - Definition}
\frametitle{Bernoulli Equation - Definition}

\begin{tcolorbox}[title=Definition, colback=blue!5!white, colframe=blue!75!black]
A Bernoulli differential equation is of the form:
\[
\frac{dy}{dx} + P(x)y = Q(x)y^n
\]
where \( n \neq 0, 1 \).
\end{tcolorbox}

\pause

To solve it:
\begin{enumerate}
    \item Divide both sides by \( y^n \)
    \item Substitute \( v = y^{1-n} \)
    \item Reduce to linear form
    \item Solve using integrating factor
\end{enumerate}

\end{frame}

% -------------------------------
\begin{frame}{Example 1}
\frametitle{Example 1}

\begin{tcolorbox}[title=Example 1, colback=green!5!white, colframe=green!75!black]
Solve:
\[
\frac{dy}{dx} + y = y^2 \sin x
\]
\end{tcolorbox}

\pause

Step 1: Identify form:
\[
P(x) = 1,\quad Q(x) = \sin x,\quad n = 2
\]

\pause

Step 2: Divide both sides by \( y^2 \):
\[
\frac{1}{y^2} \frac{dy}{dx} + \frac{1}{y} = \sin x
\]

\pause

Let \( v = \frac{1}{y} \Rightarrow \frac{dv}{dx} = -\frac{1}{y^2} \frac{dy}{dx} \)

\pause

Substitute:
\[
-\frac{dv}{dx} + v = \sin x \Rightarrow \frac{dv}{dx} - v = -\sin x
\]

\pause

Integrating factor: \( \mu(x) = e^{-x} \)

\end{frame}
\begin{frame}{Example 1(continued )}
\frametitle{Example 1(continued )}
Integrating factor: \( \mu(x) = e^{-x} \)

\pause

Multiply through:
\[
\frac{d}{dx} (v e^{-x}) = -\sin x e^{-x}
\]

\pause

Integrate:
\[
v e^{-x} = \int -\sin x e^{-x} dx
\]

\pause

Result:
\[
v = \frac{\sin x + \cos x}{2} + C e^x
\quad \Rightarrow \quad y = \frac{1}{\frac{\sin x + \cos x}{2} + C e^x}
\]

\end{frame}
% -------------------------------
\begin{frame}{Example 2}
\frametitle{Example 2}

\begin{tcolorbox}[title=Example 2, colback=yellow!5!white, colframe=orange!75!black]
Solve:
\[
\frac{dy}{dx} + \frac{2}{x} y = x^2 y^3
\]
\end{tcolorbox}

\pause

Step 1: Identify form:
\[
P(x) = \frac{2}{x},\quad Q(x) = x^2,\quad n = 3
\]

\pause

Step 2: Divide both sides by \( y^3 \):
\[
\frac{1}{y^3} \frac{dy}{dx} + \frac{2}{x y^2} = x^2
\]

\pause

Let \( v = y^{-2} \Rightarrow \frac{dv}{dx} = -2 y^{-3} \frac{dy}{dx} \)

 

\end{frame}
\begin{frame}{Example 2(continued )}
\frametitle{Example 2(continued )}
    
Solve for \( \frac{dy}{dx} \):
\[
\Rightarrow \frac{1}{y^3} \frac{dy}{dx} = -\frac{1}{2} \frac{dv}{dx}
\]

\pause

Substitute:
\[
-\frac{1}{2} \frac{dv}{dx} + \frac{2}{x} v = x^2
\Rightarrow \frac{dv}{dx} - \frac{4}{x} v = -2x^2
\]

\pause

Integrating factor:
\[
\mu(x) = e^{\int -\frac{4}{x} dx} = x^{-4}
\]

\pause

Multiply:
\[
x^{-4} \frac{dv}{dx} - \frac{4}{x^5}v = -2x^{-2}
\Rightarrow \frac{d}{dx}(v x^{-4}) = -2x^{-2}
\]

\pause

Integrate:
\[
v x^{-4} = \int -2x^{-2} dx = 2x^{-1} + C
\Rightarrow v = x^4 (2x^{-1} + C) = 2x^3 + Cx^4
\]

\pause

Recall \( v = y^{-2} \Rightarrow y = \frac{1}{\sqrt{2x^3 + Cx^4}} \)

\end{frame}
% -------------------------------
\begin{frame}{Example 3 }
\frametitle{Example 3}

\begin{tcolorbox}[title=Example 3, colback=blue!5!white, colframe=blue!75!black]
Solve:
\[
\frac{dy}{dx} - 3y = -2y^{1/2}
\]
\end{tcolorbox}

\pause

Step 1: Identify form:
\[
P(x) = -3,\quad Q(x) = -2,\quad n = \frac{1}{2}
\]

\pause

Step 2: Divide by \( y^{1/2} \):
\[
y^{-1/2} \frac{dy}{dx} - 3y^{1/2} = -2
\]

\pause

Let \( v = y^{1 - 1/2} = y^{1/2} \Rightarrow \frac{dv}{dx} = \frac{1}{2} y^{-1/2} \frac{dy}{dx} \)

\pause

\[
\Rightarrow y^{-1/2} \frac{dy}{dx} = 2 \frac{dv}{dx}
\]
\end{frame}
\begin{frame}{Example 3(continued )}
\frametitle{Example 3(continued )}
 
    Substitute:
\[
2 \frac{dv}{dx} - 3v = -2
\Rightarrow \frac{dv}{dx} - \frac{3}{2}v = -1
\]

\pause

Integrating factor:
\[
\mu(x) = e^{-\frac{3}{2}x}
\]

\pause

Multiply:
\[
\frac{d}{dx} \left( v e^{-\frac{3}{2}x} \right) = -e^{-\frac{3}{2}x}
\]

\pause

Integrate:
\[
v e^{-\frac{3}{2}x} = \int -e^{-\frac{3}{2}x} dx = \frac{2}{3} e^{-\frac{3}{2}x} + C
\Rightarrow v = \frac{2}{3} + C e^{\frac{3}{2}x}
\]

\pause

Recall \( v = y^{1/2} \Rightarrow y = \left( \frac{2}{3} + C e^{\frac{3}{2}x} \right)^2 \)

\end{frame}
%---------------------------------------------
\section{Exact Differential Equations}
\begin{frame}{Definition of Exact Equations}
    \begin{block}{General Form}
        A differential equation of the form
        \[
        M(x, y)\,dx + N(x, y)\,dy = 0
        \]
        is said to be \textbf{exact} if
        \[
        \frac{\partial M}{\partial y} = \frac{\partial N}{\partial x}
        \]
    \end{block}\pause
    \begin{block}{Concept}
        If exact, then there exists a function \( \phi(x, y) \) such that:
        \[
        d\phi = M\,dx + N\,dy
        \]
        and the solution is given by \( \phi(x, y) = C \).
    \end{block}
\end{frame}

\begin{frame}{Example 1: Exact Equation}
    \textbf{Solve:}
    \[
    (2xy + y^2)\,dx + (x^2 + 2xy)\,dy = 0
    \]\pause
    Here,
    \[
    M = 2xy + y^2, \quad N = x^2 + 2xy
    \]\pause
    Compute partial derivatives:
    \[
    \frac{\partial M}{\partial y} = 2x + 2y, \quad \frac{\partial N}{\partial x} = 2x + 2y
    \]\pause
    Since \( \frac{\partial M}{\partial y} = \frac{\partial N}{\partial x} \), the equation is \textbf{exact}.
\end{frame}

\begin{frame}{Solution of Example 1}
    Integrate \( M \) with respect to \( x \):
    \[
    \phi(x, y) = \int M\,dx = \int (2xy + y^2)\,dx = x^2y + xy^2 + h(y)
    \]\pause
    Differentiate \( \phi \) with respect to \( y \):
    \[
    \frac{\partial \phi}{\partial y} = x^2 + 2xy + h'(y)
    \]\pause
    Compare with \( N = x^2 + 2xy \), so \( h'(y) = 0 \Rightarrow h(y) = C \).\pause
    Hence,
    \[
    \phi(x, y) = x^2y + xy^2 = C
    \]
    \textbf{This is the required solution.}
\end{frame}

\begin{frame}{Example 2: Exact Equation}
    \textbf{Solve:}
    \[
    (3x^2y - y^3)\,dx + (x^3 - 3xy^2)\,dy = 0
    \]\pause
    Here,
    \[
    M = 3x^2y - y^3, \quad N = x^3 - 3xy^2
    \]\pause
    \[
    \frac{\partial M}{\partial y} = 3x^2 - 3y^2, \quad \frac{\partial N}{\partial x} = 3x^2 - 3y^2
    \]\pause
    Since these are equal, the equation is exact.\pause

    Integrate \( M \) with respect to \( x \):
    \[
    \phi(x, y) = \int (3x^2y - y^3)\,dx = x^3y - xy^3 + h(y)
    \]\pause
    Differentiate with respect to \( y \):
    \[
    \frac{\partial \phi}{\partial y} = x^3 - 3xy^2 + h'(y)
    \] 
\end{frame}

\begin{frame}{Example 2: Exact Equation[continued]}
  
    Differentiate with respect to \( y \):
    \[
    \frac{\partial \phi}{\partial y} = x^3 - 3xy^2 + h'(y)
    \]\pause
    Compare with \( N \): \( h'(y) = 0 \Rightarrow h(y) = 0 \)
    \[
    \Rightarrow x^3y - xy^3 = C
    \]
\end{frame}



%---------------------------------------------
\section{Practice Problems}
\begin{frame}{Try Yourself}
    \begin{enumerate}
        \item Solve: \( \displaystyle \frac{dy}{dx} + 2y = e^{-x} \)
        \item Test for exactness and solve: \( (2x + y)\,dx + (x + 2y)\,dy = 0 \)
        \item Solve: \( (y - 2x)\,dx + (x - 2y)\,dy = 0 \)
    \end{enumerate}
\end{frame}

%---------------------------------------------
\section{Summary}
\begin{frame}{Summary}
    \begin{itemize}
        \item Linear differential equations use the integrating factor method.
        \item Exact equations satisfy \( \frac{\partial M}{\partial y} = \frac{\partial N}{\partial x} \).
        \item Both types have structured solution techniques.
        \item These methods are essential for solving real-world physical, chemical, and engineering problems.
    \end{itemize}
\end{frame}

\end{document}
